% science_template.tex
% See accompanying readme.txt for copyright statement, change log etc.

% Any modification of this template, including writing a paper using it,
% MUST rename the file i.e. use a different file name.

%%%%%%%%%%%%%%%% START OF PREAMBLE %%%%%%%%%%%%%%%

% Basic setup. Authors shouldn't need to adjust these commands.
% It's annoying, but please do NOT strip these into a separate file.
% They need to be included in this .tex for our production software to work.

% Use the basic LaTeX article class, 12pt text
\documentclass[12pt]{article}

% Science uses Times font. If you don't have this installed (most LaTeX installations will be
% fine) or prefer the old Computer Modern fonts, comment out the following line
\usepackage{newtxtext,newtxmath}
% Depending on your LaTeX fonts installation, you might get better results with one or both of these:
%\usepackage{mathptmx}
%\usepackage{txfonts}
\usepackage{hyperref} % custom package
% Allow external graphics files
\usepackage{graphicx}

% Use US letter sized paper with 1 inch margins
\usepackage[letterpaper,margin=1in]{geometry}

% Double line spacing, including in captions
\linespread{1.5} % For some reason double spacing is 1.5, not 2.0!

% One space after each sentence
\frenchspacing

% Abstract formatting and spacing - no heading
\renewenvironment{abstract}
	{\quotation}
	{\endquotation}

% No date in the title section
\date{}

% Reference section heading
\renewcommand\refname{References and Notes}

% Figure and Table labels in bold
\makeatletter
\renewcommand{\fnum@figure}{\textbf{Figure \thefigure}}
\renewcommand{\fnum@table}{\textbf{Table \thetable}}
\makeatother

% Call the accompanying scicite.sty package.
% This formats citation numbers in Science style.
\usepackage{scicite}

% Provides the \url command, and fixes a crash if URLs or DOIs contain underscores
\usepackage{url}

%%%%%%%%%%%% CUSTOM COMMANDS AND PACKAGES %%%%%%%%%%%%

% Authors can define simple custom commands e.g. as shortcuts to save on typing
% Use \newcommand (not \def) to avoid overwriting existing commands.
% Keep them as simple as possible and note the warning in the text below.
% Example:
\newcommand{\pcc}{\,cm$^{-3}$}	% per cm-cubed

% Please DO NOT import additional external packages or .sty files.
% Those are unlikely to work with our conversion software and will cause problems later.
% Don't add any more \usepackage{} commands.


%%%%%%%%%%%%%%%% TITLE AND AUTHORS %%%%%%%%%%%%%%%%

% Title of the paper.
% Keep it short and understandable by any reader of Science.
% Avoid acronyms or jargon. Use sentence case.
\def\scititle{
	{{title}}
}
% Store the title in a variable for reuse in the supplement (otherwise \maketitle deletes it)
\title{\bfseries \boldmath \scititle}

% Author and institution list.
% % Institution numbers etc. should be hard-coded, do *not* use the \footnote command.
\author{
	
		{{author.firstname}}~{{author.lastname}}$^{ {{author.affiliation}}\ast }$\and,
	
	% You can write out first names or use initials - either way is acceptable, but be consistent
	% First~Author$^{1\ast\dagger}$,
	% A.~Scientist$^{2\dagger}$,
	% Someone~E.~Else$^{2}$\and
	% Additional lines of authors should be inserted using the \and command (not \\)
	% Institution list, in a slightly smaller font
	
	\small$^{ {{loop.index}} }${{affil}}\and
    
	% \small$^{1}$Department, Institution, City \& Postal Code, Country.\and
	% \small$^{2}$Another Department, Different Institution, City \& Postal Code, Country.\and
	% Identify at least one corresponding author, with contact emailll$^\ast$Corresponding author. Email: example@mail.com\and
	% Joint contributions can be indicated like this
	\small$^\ast$Corresponding author. Email: {{ authors[0].email }}
	% \small$^\dagger$These authors contributed equally to this work.
}
% \author{
% 	% You can write out first names or use initials - either way is acceptable, but be consistent
% 	First~Author$^{1\ast\dagger}$,
% 	A.~Scientist$^{2\dagger}$,
% 	Someone~E.~Else$^{2}$\and
% 	% Additional lines of authors should be inserted using the \and command (not \\)
% 	% Institution list, in a slightly smaller font
% 	\small$^{1}$Department, Institution, City \& Postal Code, Country.\and
% 	\small$^{2}$Another Department, Different Institution, City \& Postal Code, Country.\and
% 	% Identify at least one corresponding author, with contact email address
% 	\small$^\ast$Corresponding author. Email: example@mail.com\and
% 	% Joint contributions can be indicated like this
% 	\small$^\dagger$These authors contributed equally to this work.
% }
%%%%%%%%%%%%%%%%% END OF PREAMBLE %%%%%%%%%%%%%%%%


%%%%%%%%%%%%%%%% START OF MAIN TEXT %%%%%%%%%%%%%%%
\begin{document}

% Insert the title and author list
\maketitle

% Abstract, in bold
% There are strict length limits, and not all formats have abstracts.
% Consult the journal instructions to authors for details.
% Do not cite any references in the abstract.
\begin{abstract} \bfseries \boldmath
{{abstract | replace('\n', ' ')}}
\end{abstract}


% The first paragraph of any Science paper does NOT have a heading
% Nor is it indented
\noindent

{{body}}

% If your text is very short you might need to uncomment the following line to avoid
% layout problems with the figures and tables.
%\newpage


% %%%%%%%%%%%%%%%% MAIN TEXT FIGURES %%%%%%%%%%%%%%%

% \begin{figure} % Do NOT use \begin{figure*}
% 	\centering
% 	\includegraphics[width=0.6\textwidth]{example_figure} % for an image file named example_figure.*
% 	% Pick an appropriate width - in print, figures are usually one or two columns wide, which can
% 	% be approximated by 0.3\textwidth or 0.6\textwidth respectively. Use appropriate label sizes.

% 	% Captions go below figures
% 	\caption{\textbf{All captions must start with a short bold sentence, acting as a title.}
% 		Then explain what is being shown, the meanings of any line styles, plotting symbols etc.
% 		Multi-panel figures must label the panels A, B, C, etc. and refer to them in the caption
% 		like this: (\textbf{A}) Description of panel A. (\textbf{B}) Description of panel B.
% 		Captions are placed below figures.}
% 	\label{fig:example} % give each figure a logical label name
% \end{figure}


% %%%%%%%%%%%%%%%% MAIN TEXT TABLES %%%%%%%%%%%%%%%

% \begin{table} % Do NOT use \begin{table*}
% 	\centering
% 	% Captions go above tables
% 	\caption{\textbf{All captions must start with a short bold sentence, acting as a title.}
% 		Then explain what is being listed in the table, the meaning of each column etc.
% 		Captions are placed above tables.}
% 	\label{tab:example} % give each table a logical label name

% 	\begin{tabular}{lccc} % four columns, alignment for each
% 		\\
% 		\hline
% 		Sample & $A$ & $B$ & $C$\\
% 		 & (unit) & (unit) & (unit)\\
% 		\hline
% 		First & 1 & 2 & 3\\
% 		Second & 4 & 6 & 8\\
% 		Third & 5 & 7 & 9\\
% 		\hline
% 	\end{tabular}
% \end{table}



%%%%%%%%%%%%%%%% REFERENCES %%%%%%%%%%%%%%%

\clearpage % Clear all remaining figures and tables then start a new page

% The list of references goes after the main text and before the acknowledgements
% When preparing an initial submission, we recommend you use BibTeX, like this:
%
\bibliographystyle{sciencemag}

\bibliography{ {{bibliography}} }



% After the paper has completed peer review and been revised ready for acceptance,
% you should comment out the lines above and copy-paste the contents of your .bbl
% file here instead. This will help ensure that our conversion software works correctly.
% Remember to re-run BibTeX first - check the timestamp!
%
% Example of the first three entries copy-pasted from science_template.bbl:
%
%\begin{thebibliography}{1}
%
%\bibitem{example}
%A.~N. {Author}, An example reference. \emph{Journal of Improbable Research}
%  \textbf{1}, 67 (2020).
%
%\bibitem{example2}
%F.~M. {Surname}, S.~{Author}, A second example. \emph{Interesting Research
%  Letters} \textbf{32}, 897 (2019).
%
%\bibitem{example_preprint}
%P.~{One}, P.~{Two}, P.~{Three}, {An unpublished preprint}. \emph{preprint}
%  (2021), arXiv:2101.12345.
%
%\end{thebibliography}


%%%%%%%%%%%%%%%% ACKNOWLEDGEMENTS %%%%%%%%%%%%%%%

\section*{Acknowledgments}
{{acknowledgements}}
\paragraph*{Funding:}
{{funding}}
\paragraph*{Author contributions:}
{{authorcontribution}}
\paragraph*{Competing interests:}
{{competinginterests | default('There are no competing interests to declare.')}}
\paragraph*{Data and materials availability:}
\dataavailability{ {{dataavailability | default('please add a "data availability" section')}} } %% use this section when having only data sets available


%%%%%%%%%%%%%%%% SUPPLEMENT LIST %%%%%%%%%%%%%%%

% List the contents of your Supplementary Materials, including the numbers of any
% supplementary figures, tables, external data files etc. and any references that are
% cited only in the supplement. In this example, refs. 7-8 are cited only in the supplement.
% Fill out your numbers accordingly and delete any lines that aren't applicable.
\subsection*{Supplementary materials}
Materials and Methods\\
Supplementary Text\\
Figs. S1 to S3\\
Tables S1 to S4\\
References \textit{(7-\arabic{enumiv})}\\ % automatically fills out the last reference number
% (filling out the other numbers automatically is possible but fiddly and liable to break)
Movie S1\\
Data S1

%%%%%%%%%%%%%%%% END OF MAIN TEXT %%%%%%%%%%%%%%%

\newpage

%%%%%%%%%%%%%%%% START OF SUPPLEMENT %%%%%%%%%%%%%%%

% Figures, tables, equations and pages in the supplement are numbered S1, S2 etc.
\renewcommand{\thefigure}{S\arabic{figure}}
\renewcommand{\thetable}{S\arabic{table}}
\renewcommand{\theequation}{S\arabic{equation}}
\renewcommand{\thepage}{S\arabic{page}}
\setcounter{figure}{0}
\setcounter{table}{0}
\setcounter{equation}{0}
\setcounter{page}{1} % not 0 as \newpage already started a supplementary page
% References continue the numbering from the main text.


%%%%%%%%%%%%%%%% SUPPLEMENT TITLE PAGE %%%%%%%%%%%%%%%

\begin{center}
\section*{Supplementary Materials for\\ \scititle}

% Author list for the supplement
% Indicate the corresponding authors, but do NOT include institutions here
% It would be nice if the template auto-generated this, but doing so is complicated...
First~Author$^{\ast\dagger}$,
A.~Scientist$^\dagger$,
Someone~E.~Else\\ % we're not in a \author{} environment this time, so use \\ for a new line
\small$^\ast$Corresponding author. Email: example@mail.com\\
\small$^\dagger$These authors contributed equally to this work.
\end{center}

% Fill out the numbers for each type of supplementary material,
% and delete any lines that aren't applicable.
% These are just example numbers that don't match the rest of this template.
\subsubsection*{This PDF file includes:}
Materials and Methods\\
Supplementary Text\\
Figures S1 to S3\\
Tables S1 to S4\\
Captions for Movies S1 to S2\\
Captions for Data S1 to S2

\subsubsection*{Other Supplementary Materials for this manuscript:}
Movies S1 to S2\\
Data S1 to S2

\newpage

%%%%%%%%%%%%%%%% MATERIALS AND METHODS %%%%%%%%%%%%%%%

\subsection*{Materials and Methods}

{{materialsandmethods | default('Please add a "Materials and Methods" section.')}} % use this section when having only methods available

%%%%%%%%%%%%%%%% SUPPLEMENTARY TEXT %%%%%%%%%%%%%%%


\subsection*{Supplementary Text}
{{appendix | default('Please add an "Appendix" or "Supplementary Text" section.')}} % use this section when having only methods available


% %%%%%%%%%%% CAPTIONS FOR OTHER SUPPLEMENTARY FILES %%%%%%%%%%

% \clearpage % Clear all remaining figures and tables then start a new page

% \paragraph{Caption for Movie S1.}
% \textbf{All captions must start with a short bold sentence, acting as a title.}
% Then explain what is shown in the supplementary video file.
% Give as much detail as you would for a figure e.g. explain axes, color maps etc.
% If the video is an animated equivalent of one of the static figures, state e.g.
% `Animated version of Figure~\ref{fig:example}.'

% \paragraph{Caption for Data S1.}
% \textbf{All captions must start with a short bold sentence, acting as a title.}
% Then explain what is included in the supplementary data file.
% Give as much detail as you would for a table e.g. explain the meaning of every column,
% units used, any special notation etc.


%%%%%%%%%%%%%%%% SUPPLEMENTARY REFERENCES %%%%%%%%%%%%%%%

% Do NOT include a reference list in the supplement.
% All references must be in a single list at the end of the main text.
% The copyeditors will ensure that the correct reference list appears with each version of the paper
% (print, HTML, PDF, mobile app, metadata for bibliographic databases etc.)

\end{document}
% End of science_template.tex